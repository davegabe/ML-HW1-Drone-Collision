\section{Regression Task}
\label{sec:regression}
In this section I will describe the regression task that I have implemented, the way the dataset has been preprocessed and the results obtained with the different models.

\subsection{Data Preprocessing}
\label{sec:preprocessing}

\subsubsection{Feature selection}
\label{sec:preprocessing-feature-selection}
The first step is to select the features that will be used for the regression task.
The dataset for each drone contains the following 7 features: the position, the velocity and the target position of the drone and the angle between the drone and the target (relative to the north).
Also in this case, I have removed the angle from the dataset since it can be computed from the position and the velocity.
The dataset is then composed by $6 \times 5 = 30$ features for the 5 drones in the environment and the label, which is the minimum CPA (Closest Point of Approach) between the drones.

\subsubsection{Normalization}
\label{sec:preprocessing-normalization}
Since the dataset contains features with different ranges, it is necessary to normalize the dataset.
Also in this case, the normalization is performed for each row separately respecting the same approach described in Section~\ref{sec:preprocessing-normalization}.
In this case, the normalization is performed also for the label. The minimum CPA is a relative value, so it can used the same normalization applied to velocity.

\begin{equation}
    \norm(minCPA_{D^i}) = \frac{minCPA_{D^i}}{maxCoord - minCoord}
\end{equation}
Where:
\begin{conditions}
    maxX & $\max\limits_{1 \leq i \leq 5} x_{D^i}$\\

    maxY & $\max\limits_{1 \leq i \leq 5} y_{D^i}$\\

    minX & $\min\limits_{1 \leq i \leq 5} x_{D^i}$\\

    minY & $\min\limits_{1 \leq i \leq 5} y_{D^i}$\\

    maxCoord & $\max(maxX, maxY)$\\

    minCoord & $\min(minX, minY)$
\end{conditions}

\subsubsection{Splitting}
\label{sec:preprocessing-splitting}
The dataset has been splitted in training and test set with a ratio of 80/20.

\subsection{Training}
\label{sec:training}

\subsubsection{Regression models}
\label{sec:training-regression-models}

\subsubsection{Hyperparameter tuning}
\label{sec:training-hyperparameter-tuning}

\subsection{Evaluation}
\label{sec:evaluation}
